% !TEX root = ../matan.tex

\section{Формула Стирлинга. Неравенство Юнга. Неравенство Гёльдера}

\begin{Definition}{Формула Стирлинга}{}
	\[
	n!\sim \sqrt{2\pi n}\left(\frac{n}{e}\right)^n\left(1+O\!\left(\frac1n\right)\right),
	\qquad n\to\infty.
	\]
\end{Definition}

\begin{proof}
	Пишем $\log\equiv\ln$. Тогда
	\[
	\ln(n!)=\ln 1+\ln 2+\dots+\ln n=\sum_{k=1}^n \ln k.
	\]
	
	\begin{center}
		\begin{tikzpicture}[x=1.1cm,y=3.2cm,>=Stealth]
			
			\pgfmathsetmacro{\n}{6}
			\pgfmathsetmacro{\nminus}{\n-1}
			\pgfmathsetmacro{\nplus}{\n+1}
			\pgfmathsetmacro{\yn}{ln(\n)/ln(10)}        
			\pgfmathsetmacro{\yplus}{ln(\nplus)/ln(10)}
			
			\draw[step=0.5,gray!25,very thin] (-0.6,-1.05) grid (8.6,1.15);
			
			\draw[thick,-{Stealth[length=3mm]}] (-0.6,0) -- (8.6,0);
			\draw[thick,-{Stealth[length=3mm]}] (0,-1.05) -- (0,1.15); 
			
			\foreach \x/\lab in {1/1,2/2,3/3,\nminus/{$n-1$},\n/{$n$},\nplus/{$n+1$}}{
				\draw[thick] (\x,0.03) -- (\x,-0.03);
				\node[below] at (\x,0) {\lab};
			}
			\node[below] at (4.1,0) {$\dots$};
			
			\draw[thick,domain=0.15:8.2,samples=240,smooth]
			plot (\x,{ln(\x)/ln(10)});
			
			\fill (1,0) circle (1.6pt);
			
			\fill[pattern=north east lines,pattern color=violet,opacity=0.35]
			(\nminus,0) rectangle (\n,\yn);
			\draw[very thick,violet]
			(\nminus,0) rectangle (\n,\yn);
			
			\node[violet] at ({(\nminus+\n)/2},{\yn/2}) {$\ln n\cdot1$};
			
			\fill[pattern=north east lines,pattern color=blue,opacity=0.25]
			(\n,0) rectangle (\nplus,\yplus);
			\draw[thick,dashed]
			(\n,0) rectangle (\nplus,\yplus);
			
			\draw[very thick,violet]
			(\nminus,\yn) -- (\n,\yn) -- (\n,\yplus) -- (\nplus,\yplus);
			
			\draw[very thick,violet,domain=\nminus:\nplus,samples=120,smooth]
			plot (\x,{ln(\x)/ln(10)});
			
			\draw[-{Stealth[length=2mm]}] (8.1,0.98) -- (7.2,{ln(7.2)/ln(10)});
			\node[right] at (8.1,0.98) {$\ln x$};
			
		\end{tikzpicture}
	\end{center}
	
	Так как $\ln x$ возрастает, то для любого $n\ge1$:
	\[
	\int_{n}^{n+1}\ln x\,dx \;>\; \ln n \;>\; \int_{n-1}^{n}\ln x\,dx .
	\]
	
	Суммируя по $n=1,\dots,N$
	\[
	\int_{0}^{N}\ln x\,dx \;<\; \ln(N!) \;<\; \int_{1}^{N+1}\ln x\,dx .
	\]
	
	Так как
	\[
	(x\ln x - x)' = \ln x
	\qquad\Rightarrow\qquad
	\int \ln x\,dx = x\ln x - x + \mathrm{C},
	\]
	то
	\[
	\int_{0}^{n}\ln x\,dx = n\ln n-n,
	\qquad
	\int_{1}^{n+1}\ln x\,dx = ((n+1)\ln(n+1) - (n+1)) - (1 \cdot \ln1 - 1).
	\]
	Следовательно,
	\[
	n\ln n - n \;<\; \ln(n!) \;<\; (n+1)\ln(n+1) - n,
	\]
	и, возводя в экспоненту,
	\[
	n^n e^{-n} \;<\; n! \;<\; (n+1)^{n+1} e^{-n}.
	\]
	
	Определим
	\[
	d_n := \ln(n!) - \Bigl(\bigl(n+\tfrac12\bigr)\ln n - n\Bigr).
	\]

	Если $d_n\to C$, то
		\[
		n! = e^{d_n}\, n^{n+\frac12}e^{-n}
		= e^{C}\sqrt{n}\left(\frac{n}{e}\right)^n(1+o(1)).
		\]
	Кроме того, оценка $d_n=C+O\left(\tfrac1n\right)$ даст именно множитель $1+O\left(\tfrac1n\right)$.
	
	Вычислим разность:
	\begin{align*}
		d_n-d_{n+1}
		&= \ln(n!) - \Bigl(n+\tfrac12\Bigr)\ln n + n
		- \ln\bigl((n+1)!\bigr) + \Bigl(n+1+\tfrac12\Bigr)\ln(n+1) - (n+1) \\
		&= -\ln(n+1)
		- \Bigl(n+\tfrac12\Bigr)\ln n
		+ \Bigl(n+\tfrac32\Bigr)\ln(n+1) - 1 \\
		&= \Bigl(n+\tfrac12\Bigr)\bigl(\ln(n+1)-\ln n\bigr)-1 \\
		&= \Bigl(n+\tfrac12\Bigr)\ln\!\Bigl(\frac{n+1}{n}\Bigr)-1.
	\end{align*}
	
	Положим
	\[
	\frac{n+1}{n}=1+\frac1n
	= \frac{1+\frac1{2n+1}}{1-\frac1{2n+1}}
	= \frac{1+t}{1-t},
	\qquad t:=\frac{1}{2n+1}.
	\]
	\emph{(Эта форма удобна, потому что $\ln\frac{1+t}{1-t}$ имеет красивый ряд.)}
	
	Для $|t|<1$ имеем сходящийся степенной ряд:
	\[
	\ln\!\Bigl(\frac{1+t}{1-t}\Bigr)
	= \ln(1+t)-\ln(1-t)
	= 2\left(t+\frac{t^3}{3}+\frac{t^5}{5}+\cdots\right).
	\]
	Обозначим остаток частичной суммы:
	\[
	\ln\!\Bigl(\frac{1+t}{1-t}\Bigr)
	= 2\left(t+\frac{t^3}{3}+\cdots+\frac{t^{2k+1}}{2k+1}\right)+r_{2k+1}(t),
	\qquad |t|<1,
	\]
	где при фиксированном $t$ выполнено $r_{2k+1}(t)\to0$ при $k\to\infty$.
	
	Подставим это в $d_n-d_{n+1}$ и используем $n+\tfrac12=\frac{1}{2t}$:
	\begin{align*}
		d_n-d_{n+1}
		&=\Bigl(n+\tfrac12\Bigr)\ln\!\Bigl(\frac{1+t}{1-t}\Bigr)-1 \\
		&= \frac{1}{2t}\left(2\left(t+\frac{t^3}{3}+\cdots+\frac{t^{2k+1}}{2k+1}\right)+r_{2k+1}(t)\right)-1 \\
		&=\left(1+\frac{t^2}{3}+\frac{t^4}{5}+\cdots+\frac{t^{2k}}{2k+1}\right)+\frac{r_{2k+1}(t)}{2t}-1 \\
		&=\frac{t^2}{3}+\frac{t^4}{5}+\cdots+\frac{t^{2k}}{2k+1}+\frac{r_{2k+1}(t)}{2t}.
	\end{align*}
	Отсюда сразу видно, что $d_n-d_{n+1}>0$, то есть $(d_n)$ убывает.
	
	Теперь оценим сверху. Так как для $j\ge1$ выполнено $\frac{1}{2j+1}\le \frac{1}{3}$, то
	\[
	\frac{t^2}{3}+\frac{t^4}{5}+\cdots+\frac{t^{2k}}{2k+1}
	\le \frac13\left(t^2+t^4+\cdots+t^{2k}\right).
	\]
	Переходя к пределу по $k\to\infty$ и используя сходимость ряда при $|t|<1$, получаем
	\[
	0<d_n-d_{n+1}
	\le \frac13\sum_{j=1}^{\infty} t^{2j}
	=\frac13\cdot \frac{t^2}{1-t^2}.
	\]
	
	Подставим $t=\frac{1}{2n+1}$:
	\[
	0<d_n-d_{n+1}
	\le \frac13\cdot \frac{\frac{1}{(2n+1)^2}}{1-\frac{1}{(2n+1)^2}}
	=\frac{1}{3\bigl((2n+1)^2-1\bigr)}
	=\frac{1}{12n(n+1)}
	=\frac{1}{12}\left(\frac1n-\frac1{n+1}\right).
	\]
	
	Суммируя по $j=n,\dots,n+m-1$, получаем для любого $m\ge1$:
	\[
	0<d_n-d_{n+m}
	=\sum_{j=n}^{n+m-1}(d_j-d_{j+1})
	<\frac{1}{12}\left(\frac1n-\frac1{n+m}\right).
	\]
	\emph{(Значит $d_n$ убывает и ограничена снизу, следовательно имеет предел.)}
	
	Обозначим $C:=\lim\limits_{n\to\infty} d_n$. Тогда, переходя в предыдущем неравенстве к пределу при $m\to\infty$, получаем
	\[
	0<d_n-C\le \frac{1}{12n},
	\]
	то есть
	\[
	d_n=C+O\!\left(\frac1n\right).
	\]
	Следовательно,
	\[
	\ln(n!)=\Bigl(n+\tfrac12\Bigr)\ln n-n+C+O\!\left(\frac1n\right),
	\]
	и после экспоненты:
	\[
	n!=e^{C}\sqrt{n}\left(\frac{n}{e}\right)^n\left(1+O\!\left(\frac1n\right)\right).
	\]
	Осталось найти $e^{C}$.
	
	\medskip
	\textbf{Формула Валлиса:}
	\[
	\left(\frac{(2n)!!}{(2n-1)!!}\right)^2\cdot\frac{1}{2n+1}
	\xrightarrow[n\to\infty]{}\frac{\pi}{2}.
	\]
	Отсюда
	\[
	\ln\left(\left(\frac{(2n)!!}{(2n-1)!!}\right)^2\cdot\frac{1}{2n+1}\right)
	\xrightarrow[n\to\infty]{}\ln\frac{\pi}{2}.
	\]
	
	Используем тождества
	\[
	(2n)!!=2^n n!,
	\qquad
	(2n)!=(2n)!!(2n-1)!!,
	\]
	поэтому
	\[
	\frac{(2n)!!}{(2n-1)!!}=\frac{((2n)!!)^2}{(2n)!}
	=\frac{2^{2n}(n!)^2}{(2n)!}.
	\]
	Тогда
	\[
	\ln\left(\left(\frac{2^{2n}(n!)^2}{(2n)!}\right)^2\frac{1}{2n+1}\right)
	\xrightarrow[n\to\infty]{}\ln\frac{\pi}{2}.
	\]
	
	Теперь подставим разложения через $d_n$:
	\[
	\ln(n!)=\Bigl(n+\tfrac12\Bigr)\ln n-n+d_n,\qquad
	\ln((2n)!)=\Bigl(2n+\tfrac12\Bigr)\ln(2n)-2n+d_{2n}.
	\]
	Тогда
	\begin{align*}
		\ln\left(\left(\frac{2^{2n}(n!)^2}{(2n)!}\right)^2\frac{1}{2n+1}\right)
		&=4n\ln2+4\ln(n!)-2\ln((2n)!)-\ln(2n+1)\\
		&=4d_n-2d_{2n}+\ln n-\ln2-\ln(2n+1).
	\end{align*}
	\emph{(Все главные члены порядка $n\ln n$ и $n$ сократились; остались только $d_n$ и константы.)}
	
	Переходя к пределу при $n\to\infty$ (учитывая $d_n\to C$ и $d_{2n}\to C$), получаем
	\[
	\ln\frac{\pi}{2}=4C-2C+\lim_{n\to\infty}\bigl(\ln n-\ln2-\ln(2n+1)\bigr)
	=2C-2\ln2.
	\]
	Значит
	\[
	2C-2\ln2=\ln\frac{\pi}{2}
	\quad\Rightarrow\quad
	2C=\ln(2\pi)
	\quad\Rightarrow\quad
	C=\frac12\ln(2\pi),
	\qquad
	e^{C}=\sqrt{2\pi}.
	\]
	
	Итак,
	\[
	n!=\sqrt{2\pi n}\left(\frac{n}{e}\right)^n\left(1+O\!\left(\frac1n\right)\right),
	\qquad n\to\infty.
	\]
\end{proof}

\subsection{Несколько интегральных неравенств}

\begin{Statement}{Неравенство Юнга}{}
	Пусть $a,b>0$, $p,q>0$, $\dfrac{1}{p}+\dfrac{1}{q}=1$, $p,q\ne 1$. Тогда
	\[
	a^{\frac{1}{p}}b^{\frac{1}{q}}\le \frac{a}{p}+\frac{b}{q},\qquad p>1,
	\]
	\[
	a^{\frac{1}{p}}b^{\frac{1}{q}}\ge \frac{a}{p}+\frac{b}{q},\qquad p<1.
	\]
\end{Statement}	

\begin{proof} 
	
	\textbf{Claim.} При $x>0$ верно
	\[
	x^{\alpha}-\alpha x+ \alpha-1\le 0,\qquad 0<\alpha<1,
	\]
	\[
	x^{\alpha}-\alpha x+ \alpha-1\ge 0,\qquad \alpha<0\ \ \text{или}\ \ \alpha>1.
	\]
	
	\textbf{Проверяем}
	\[
	(x^{\alpha}- \alpha x+ \alpha-1)'= \alpha\,x^{\alpha-1}- \alpha = \alpha \bigl(x^{\alpha-1}-1\bigr)=0
	\iff x=1.
	\]
	\[
	1^{\alpha}-\alpha \cdot 1+\alpha-1=0.
	\]
	Отсюда: максимум при $0<\alpha<1$, минимум при $\alpha<0$ или $\alpha>1$.
	
	\medskip
	\textbf{Положим} $x:=\dfrac{a}{b}$, \quad $\alpha:=\dfrac{1}{p}$, \quad
	$\dfrac{1}{q}:=1-\dfrac{1}{p}=1-\alpha$.
	
	\textbf{Тогда}
	\[
	\left(\frac{a}{b}\right)^{\!\frac1p}-\frac{1}{p}\cdot\frac{a}{b}+\frac{1}{p}-1\le 0,
	\qquad 0<\frac{1}{p}<1 \ \ (\Leftrightarrow\ p>1).
	\]
	Умножая на $b>0$, получаем
	\[
	a^{\frac1p}\,b^{\,1-\frac1p}\le \frac{a}{p}+b\Bigl(1-\frac{1}{p}\Bigr).
	\]
	Так как $\dfrac{1}{q}=1-\dfrac{1}{p}$, то
	\[
	a^{\frac1p}b^{\frac1q}\le \frac{a}{p}+\frac{b}{q}.
	\]
	
	А при $\dfrac{1}{p}<0$ или $\dfrac{1}{p}>1$ (то есть $p<1$) получаем обратное неравенство:
	\[
	a^{\frac1p}b^{\frac1q}\ge \frac{a}{p}+\frac{b}{q}.
	\]
\end{proof}

\begin{Statement}[breakable=false]{Неравенство Гёльдера}{}
	Пусть $f,g:(a,b)\to\mathbb{R}$,
	и $f,g$ интегрируемы по Риману на $[c,d]\subset(a,b)$ для любого $[c,d]$.
	
	\medskip
	\textbf{Тогда:}
	\[
	\int_a^b \bigl|f(x)\,g(x)\bigr|\,dx
	\le
	\left(\int_a^b |f(x)|^{p}\,dx\right)^{\!\frac1p}
	\cdot
	\left(\int_a^b |g(x)|^{q}\,dx\right)^{\!\frac1q},
	\]
	\[
	\forall\, p,q>0,\ p,q<\infty:\qquad \frac1p+\frac1q=1.
	\]
	
	\medskip
	Если $p=q=2$, то имеем неравенство КБШ:
	\[
	\int_a^b |f g|
	\le
	\sqrt{\int_a^b |f|^{2}}\;\cdot\;\sqrt{\int_a^b |g|^{2}}.
	\]
\end{Statement}	
