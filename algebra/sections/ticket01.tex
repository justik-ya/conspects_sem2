% !TEX root = ../algebra_02.tex

\section{Что то будет}
% !TEX root = ../линейная_алгебра.tex

\section{Ранг матрицы над полем. Элементарные преобразования. Теорема Кронекера--Капелли. Прямые суммы}

\subsection{Ранг матрицы над произвольным полем}

Пусть $k$ — поле, $A \in k^{m \times n}$.

\begin{Definition}{Ранг матрицы}{}
    \emph{Рангом} матрицы $A$ называется число линейно независимых столбцов:
    \[
        \mathrm{rk}(A) = \mathrm{rk}(A[{:},1],\, \ldots,\, A[{:},n]).
    \]
\end{Definition}

\begin{Theorem}{Равенство строчного и столбцового рангов}{}
    Для любой матрицы $A \in k^{m \times n}$:
    \[
        \mathrm{rk}(A[1,{:}],\, \ldots,\, A[m,{:}]) = \mathrm{rk}(A^T).
    \]
    В частности, $\mathrm{rk}(A) = \mathrm{rk}(A^T)$.
\end{Theorem}

\subsection{Инвариантность ранга при элементарных преобразованиях строк}

\begin{Theorem}{Инвариантность ранга}{}
    Ранг $\mathrm{rk}(A)$ и ранг $\mathrm{rk}(A^T)$ не меняются при элементарных преобразованиях строк матрицы $A$.
\end{Theorem}

\begin{proof}
Достаточно проверить для одного элементарного преобразования $A \to A'$.

\textbf{Шаг 1 ($\mathrm{rk}(A') \leq \mathrm{rk}(A)$).} При элементарном преобразовании строк существует обратимая матрица $U \in M_m(k)$ такая, что $A' = UA$. Поэтому каждый столбец $A'[{:},j]$ лежит в линейной оболочке столбцов $A$:
\[
    \mathrm{Lin}(A'[{:},1],\, \ldots,\, A'[{:},n]) \subseteq \mathrm{Lin}(A[{:},1],\, \ldots,\, A[{:},n]),
\]
откуда $\mathrm{rk}((A')^T) \leq \mathrm{rk}(A^T)$.

\textbf{Шаг 2 ($\mathrm{rk}(A) \leq \mathrm{rk}(A')$).} Так как $U$ обратима, существует обратное преобразование $A \xrightarrow{\text{эл. пр.}} A'$, откуда симметрично $\mathrm{rk}(A^T) \leq \mathrm{rk}((A')^T)$.

Объединяя оба шага: $\mathrm{rk}(A) = \mathrm{rk}(A')$.
\end{proof}

\subsubsection{Детали доказательства: ядра и линейные оболочки}

Пусть $\mathrm{rk}(A) = r$, то есть существуют индексы $j_1, \ldots, j_r$ такие, что $A[{:},j_1], \ldots, A[{:},j_r]$ линейно независимы (ЛНЗ). Рассмотрим $A' = A[{:},\{j_1,\ldots,j_r\}] \in k^{m \times r}$ — подматрицу из этих $r$ столбцов.

Для вектора $x$ с компонентой $d_{i_1}$ на месте $i_1$ и нулями в остальных местах:
\[
    A \begin{pmatrix} 0 \\ d_{i_1} \\ \vdots \\ 0 \end{pmatrix} = 0 \implies A' \cdot d_{i_1} \cdot e_{i_1} = 0,
\]
откуда $d_{i_1} = 0$. Следовательно, существование ненулевого $d_{i_1}, \ldots, d_{i_\ell}$ невозможно, что и означает ЛНЗ.

Так как $A' = UA$, то $UA\begin{pmatrix}0\\d_{i_1}\\\vdots\\d_{i_2}\end{pmatrix} = 0$, то есть $A'[{:},i_1], \ldots, A'[{:},i_\ell]$ — ЛНЗ.

При $\mathrm{rk}(A) = r$: для любого набора из $r+1$ столбцов матрица $A$ — ЛЗ $\Rightarrow$ матрица $A'$ тоже ЛЗ, откуда $\mathrm{rk}(A') \leq \mathrm{rk}(A)$.

Аналогично $\mathrm{rk}(A) \leq \mathrm{rk}(A')$, итого $\mathrm{rk}(A) = \mathrm{rk}(A')$.

\subsection{Неравенства для рангов суммы и произведения}

\begin{Theorem}{Неравенства для рангов}{}
    Для матриц $A \in k^{m \times n}$, $B \in k^{n \times p}$:
    \begin{enumerate}
        \item $\mathrm{rk}(A + B) \leq \mathrm{rk}(A) + \mathrm{rk}(B)$.
        \item $\mathrm{rk}(AB) \leq \min(\mathrm{rk}(A),\, \mathrm{rk}(B))$.
    \end{enumerate}
\end{Theorem}

\begin{proof}
\textbf{Неравенство 2.} Элемент $(AB)[i,j] = \sum_\nu A[i,\nu] \cdot B[\nu,j]$, поэтому
\[
    (AB)[{:},j] = \sum_\nu B[\nu,j] \cdot A[{:},\nu] \in \mathrm{Lin}(A[{:},1], \ldots, A[{:},n]).
\]
Следовательно, $\mathrm{Lin}((AB)[{:},1], \ldots, (AB)[{:},p]) \subseteq \mathrm{Lin}(A[{:},1], \ldots, A[{:},n])$, откуда $\mathrm{rk}(AB) \leq \mathrm{rk}(A)$.

Оценка $\mathrm{rk}(AB) \leq \mathrm{rk}(B)$ следует из:
\[
    \mathrm{rk}(AB) = \mathrm{rk}((AB)^T) = \mathrm{rk}(B^T A^T) \leq \mathrm{rk}(B^T) = \mathrm{rk}(B).
\]
\end{proof}

\subsubsection{Следствие: умножение на обратимую матрицу не меняет ранг}

\begin{Theorem}{Ранг при умножении на обратимую матрицу}{}
    Пусть $A \in k^{m \times n}$.
    \begin{enumerate}
        \item Если $U \in GL_m(k)$, то $\mathrm{rk}(UA) = \mathrm{rk}(A)$.
        \item Если $U \in GL_n(k)$, то $\mathrm{rk}(AU) = \mathrm{rk}(A)$.
    \end{enumerate}
\end{Theorem}

\begin{proof}
Для случая 1: $\mathrm{rk}(UA) \leq \mathrm{rk}(A)$ из неравенства выше. С другой стороны, $A = U^{-1}(UA)$, откуда $\mathrm{rk}(A) \leq \mathrm{rk}(UA)$.
\end{proof}

\begin{Remark}{}{}
    Предложение: если $U \in M_n(k)$, то
    \[
        U \in GL_n(k) \iff \mathrm{rk}(U) = n.
    \]
    Доказательство: $(\Rightarrow)$: $U = U \cdot E_n \Rightarrow \mathrm{rk}(U) = \mathrm{rk}(E_n) = n$. $(\Leftarrow)$: $U \sim D = \begin{pmatrix} E_r & 0 \\ 0 & 0 \end{pmatrix}$, $r = \mathrm{rk}(U) = n \Rightarrow U \in GL_n(k)$.
\end{Remark}

\subsection{Тензорный ранг}

Пусть $A \in k^{m \times n}$.

\begin{Definition}{Тензорный ранг}{}
    \emph{Тензорным рангом} матрицы $A$ называется
    \[
        \min \bigl\{ r \mid \exists\, A_1, \ldots, A_r,\; A = A_1 + \cdots + A_r,\; A_i \text{ — матрица ранга } 1 \bigr\}.
    \]
\end{Definition}

\begin{Theorem}{Тензорный ранг равен рангу}{}
    Тензорный ранг матрицы $A$ равен $\mathrm{rk}(A)$.
\end{Theorem}

\begin{proof}
Матрица $A$ имеет ненулевой минор порядка $r$, но нет ненулевого минора порядка $r+1$.

Доказательство использует то, что существует подматрица $B = A[\{i_1,\ldots,i_s\},\{j_1,\ldots,j_s\}]$ ранга $s$ — один из миноров порядка $r$. Столбцы $A[{:},j_1], \ldots, A[{:},j_s]$ — ЛНЗ. Столбцы $A[{:},j_{s+1}], \ldots$ — ЛЗ над ними.

Если выкинуть какой-либо столбец из ЛНЗ системы строк, ЛНЗ остальных строк сохраняется. Следовательно, $\mathrm{rk}\,B < s$ — одна из строк выражается через остальные. Итого тензорный ранг $= \mathrm{rk}(A)$.
\end{proof}

\subsection{Теорема Кронекера--Капелли}

\begin{Theorem}{Кронекера--Капелли}{}
    Система линейных уравнений $(A \mid b)$ совместна тогда и только тогда, когда
    \[
        \mathrm{rk}(A \mid b) = \mathrm{rk}(A).
    \]
\end{Theorem}

\begin{proof}
Пусть $A \in k^{m \times n}$, $b \in k^m$.

$(\Rightarrow)$: Система совместна, то есть существует $\begin{pmatrix} d_1 \\ \vdots \\ d_n \end{pmatrix}$ такое, что $A \begin{pmatrix} d_1 \\ \vdots \\ d_n \end{pmatrix} = b$. Это означает $d_1 A[{:},1] + \cdots + d_n A[{:},n] = b$, то есть $b \in \mathrm{Lin}(A[{:},1], \ldots, A[{:},n]) =: W$. Следовательно,
\[
    \mathrm{Lin}(A[{:},1], \ldots, A[{:},n], b) = W,
\]
откуда $\mathrm{rk}(A \mid b) = \dim W = \mathrm{rk}(A)$.

$(\Leftarrow)$: $\mathrm{rk}(A \mid b) = \mathrm{rk}(A) \Rightarrow \mathrm{Lin}(A[{:},1], \ldots, A[{:},n], b) = W$ (м.к. $\supseteq$). Поэтому $b \in W$, то есть $b = A\begin{pmatrix} d_1 \\ \vdots \\ d_n \end{pmatrix}$ для некоторых $d_1, \ldots, d_n \in k$.
\end{proof}

\subsection{Прямые суммы подпространств}

Пусть $V$ — векторное пространство над $k$, $W_1, \ldots, W_k \leq V$ — подпространства.

\begin{Definition}{Прямая сумма}{}
    Говорят, что $V$ \emph{раскладывается во внутреннюю прямую сумму} $W_1, \ldots, W_k$, и пишут $V = W_1 \oplus \cdots \oplus W_k$, если для каждого $v \in V$ существует единственное представление
    \[
        v = w_1 + \cdots + w_k, \quad w_i \in W_i.
    \]
\end{Definition}

\begin{Remark}{}{}
    Если существует базис $e_1, \ldots, e_k$ пространства $V$, то $e_1, \ldots, e_k$ — базис $V$, откуда $V = \mathrm{Lin}(e_1) \oplus \cdots \oplus \mathrm{Lin}(e_k)$.
\end{Remark}

\subsubsection{Критерий прямой суммы двух подпространств}

\begin{Theorem}{Критерий для двух слагаемых}{}
    Для подпространств $W_1, W_2 \leq V$:
    \[
        W_1 \oplus W_2 \iff \begin{cases} V = W_1 + W_2, \\ W_1 \cap W_2 = \{0\}. \end{cases}
    \]
\end{Theorem}

\begin{proof}
$(\Rightarrow)$: Первое условие очевидно. Для второго: пусть $w \in W_1 \cap W_2$, тогда $v = w + 0 = 0 + w$, что противоречит единственности (при $w \neq 0$), значит $w = 0$.

$(\Leftarrow)$: Пусть $v = w_1 + w_2 = w_1' + w_2'$, тогда $w_1 - w_1' = w_2' - w_2$. Левая часть лежит в $W_1$, правая — в $W_2$, поэтому оба равны нулю, откуда $w_1 = w_1'$, $w_2 = w_2'$.
\end{proof}

\subsubsection{Критерий прямой суммы нескольких подпространств}

\begin{Theorem}{Критерий для нескольких слагаемых}{}
    $V = W_1 \oplus \cdots \oplus W_k$ тогда и только тогда, когда:
    \begin{enumerate}
        \item $W_1 + \cdots + W_k = V$;
        \item для каждого $j$: $W_j \cap (W_1 + \cdots + \widehat{W_j} + \cdots + W_k) = \{0\}$.
    \end{enumerate}
\end{Theorem}

\begin{Theorem}{Базис прямой суммы}{}
    Пусть $W_1, \ldots, W_k$ — подпространства $V$, $e_{i,1}, \ldots, e_{i,d_i}$ — базис $W_i$ при $i = 1, \ldots, k$. Тогда $V = W_1 \oplus \cdots \oplus W_k$ тогда и только тогда, когда
    \[
        \{e_{i,j} \mid 1 \leq i \leq k,\; 1 \leq j \leq d_i\}
    \]
    является базисом $V$.
\end{Theorem}

\begin{proof}
$(\Rightarrow)$: Для любого $v \in V$: $v = w_1 + \cdots + w_k = \sum_{i=1}^{k} \sum_{j=1}^{d_i} \alpha_{i,j}\, e_{i,j}$.

Единственность: если $\sum_{i,j} \alpha_{i,j}\, e_{i,j} = \sum_{i,j} \beta_{i,j}\, e_{i,j}$, то
\[
    \sum_i \Bigl(\sum_j \alpha_{i,j}\, e_{i,j}\Bigr) = \sum_i \Bigl(\sum_j \beta_{i,j}\, e_{i,j}\Bigr).
\]
Обозначим $w_i = \sum_j \alpha_{i,j} e_{i,j} \in W_i$ и $w_i' = \sum_j \beta_{i,j} e_{i,j} \in W_i$. Из $\sum_i w_i = \sum_i w_i'$ и единственности разложения $v = \sum_i w_i$ получаем $w_i = w_i'$ для всех $i$, откуда $\alpha_{i,j} = \beta_{i,j}$ для всех $i, j$. Таким образом, $\{e_{i,j}\}$ — базис $V$.
\end{proof}