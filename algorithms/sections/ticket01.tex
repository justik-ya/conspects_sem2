% !TEX root = ../algorithms.tex

\section{Быстрое преобразование Фурье (FFT).}

\subsection*{Свёртка и умножение многочленов}

\begin{Definition}{}{}
	Пусть заданы последовательности чисел
	$a_0,\dots,a_{n-1}$ и $b_0,\dots,b_{n-1}$.
	Определим последовательность $c_0,\dots,c_{2n-2}$ по формуле
	\[
	c_k=\sum_{t=0}^{n-1} a_t\, b_{k-t}, \qquad k=0,\dots,2n-2,
	\]
	(считаем $a_i=b_i=0$ при $i\notin[0,n-1]$).
	Эта операция называется \textbf{инволюцией (сверткой)}:
	\[
	c=a*b.
	\]
\end{Definition}

\subsubsection*{Умножение многочленов}

	Рассмотрим многочлены:
	\[
	A(x)=a_0+a_1x+\dots+a_{n-1}x^{n-1},
	\]
	\[
	B(x)=b_0+b_1x+\dots+b_{n-1}x^{n-1}.
	\]
	
	Тогда их произведение:
	\[
	C(x)=A(x)B(x)
	= c_0+c_1x+\dots+c_{2n-2}x^{2n-2},
	\]
	где коэффициенты $c_k$ задаются инволюцией.

\subsubsection*{Сложность}

\begin{itemize}
	\item Наивное умножение: $O(n^2)$.
	\item Карацуба: $O\!\left(n^{\log_2 3}\right)$.
	\item FFT (быстрое преобразование Фурье): $O(n\log n)$.
\end{itemize}

\subsection*{Идея FFT-умножения}

Хотим быстро вычислить свёртку (а значит и произведение многочленов).
Для этого:

\begin{enumerate}
	\item Выбираем число точек $N$ так, чтобы $N\ge 2n-1$ и обычно $N=2^t$.
	Коэффициенты $A,B$ дополняем нулями до длины $N$.
	\item Быстро считаем значения в $N$ точках:
	\[
	A(\omega_k),\quad B(\omega_k), \qquad k=0,\dots,N-1,
	\]
	где $\omega$ — примитивный $N$-й корень из единицы.
	\item Перемножаем покомпонентно:
	\[
	y_k = A(\omega_k)\,B(\omega_k).
	\]
	\item По значениям $y_k$ восстанавливаем коэффициенты $C$ обратным FFT.
\end{enumerate}

Ключевой факт: можно выбрать точки $\omega_k$ так, что и «вычисление значений»,
и «восстановление коэффициентов» делаются за $O(N\log N)$.

\subsection*{Реализация FFT}

Далее считаем, что $N=2^t$. Если исходная длина не степень двойки,
дополняем нулями.

\subsubsection*{Прямое FFT: из коэффициентов в значения}

Выбираем
\[
\omega = e^{2\pi i/N}=\cos\frac{2\pi}{N}+i\sin\frac{2\pi}{N},
\qquad
\omega_k=\omega^k.
\]
Нужно вычислить
\[
y_k = A(\omega^k)=\sum_{j=0}^{N-1} a_j(\omega^k)^j,\qquad k=0,\dots,N-1.
\]

Разобьём $A$ на чётные и нечётные коэффициенты:
\[
A_0(x)=a_0+a_2x+\cdots+a_{N-2}x^{\frac{N}{2}-1},
\qquad
A_1(x)=a_1+a_3x+\cdots+a_{N-1}x^{\frac{N}{2}-1}.
\]
Тогда
\[
A(x)=A_0(x^2)+xA_1(x^2).
\]
Подставим $x=\omega^k$:
\[
A(\omega^k)=A_0(\omega^{2k})+\omega^k A_1(\omega^{2k}).
\]
Обозначим
\[
y_k^{(0)}=A_0(\omega^{2k}),\qquad y_k^{(1)}=A_1(\omega^{2k}).
\]
Так как значения $\omega^{2k}$ пробегают только $N/2$ различных точек,
достаточно посчитать $y_k^{(0)},y_k^{(1)}$ для $k=0,\dots,\frac{N}{2}-1$ (рекурсивно),
а затем «склеить» ответы:
\[
y_k = y_k^{(0)} + \omega^k\, y_k^{(1)},
\qquad
y_{k+\frac{N}{2}} = y_k^{(0)} - \omega^k\, y_k^{(1)}.
\]
(Здесь использовано $\omega^{k+N/2}=-\omega^k$.)

\subsubsection*{Оценка времени прямого FFT}

На каждом уровне рекурсии решаем 2 подзадачи размера $N/2$
и делаем склейку за $O(N)$:
\[
T(N)=2T\!\left(\frac{N}{2}\right)+O(N)=O(N\log N).
\]

\subsubsection*{Обратное FFT: из значений в коэффициенты}

Прямое преобразование можно записать матрично:
\[
\begin{pmatrix}
	1 & 1 & 1 & \cdots & 1 \\
	1 & \omega^1 & (\omega^1)^2 & \cdots & (\omega^1)^{N-1} \\
	1 & \omega^2 & (\omega^2)^2 & \cdots & (\omega^2)^{N-1} \\
	\vdots & \vdots & \vdots & \ddots & \vdots \\
	1 & \omega^{N-1} & (\omega^{N-1})^2 & \cdots & (\omega^{N-1})^{N-1}
\end{pmatrix}
\begin{pmatrix}
	a_0\\ a_1\\ \vdots\\ a_{N-1}
\end{pmatrix}
=
\begin{pmatrix}
	y_0\\ y_1\\ \vdots\\ y_{N-1}
\end{pmatrix}.
\]
Это матрица Вандермонда $F_\omega$.

\begin{Statement}{}{}
	Выполнено
	\[
	F_{\omega^{-1}}\,F_\omega = N\cdot \mathbf{I}.
	\]
\end{Statement}

\begin{proof}
	Элемент в позиции $(i,j)$ равен
	\[
	\sum_{k=0}^{N-1}(\omega^i)^k(\omega^{-j})^k
	=\sum_{k=0}^{N-1}(\omega^{i-j})^k.
	\]
	Если $i\neq j$, то $q:=\omega^{i-j}\neq 1$, и по формуле суммы геометрической прогрессии
	\[
	\sum_{k=0}^{N-1} q^k=\frac{1-q^N}{1-q}
	\]
	получаем
	\[
	\sum_{k=0}^{N-1}(\omega^{i-j})^k
	=\frac{1-(\omega^{i-j})^N}{1-\omega^{i-j}}
	=\frac{1-(\omega^N)^{\,i-j}}{1-\omega^{i-j}}
	=\frac{1-1}{1-\omega^{i-j}}=0.
	\]
	Если $i=j$, то
	\[
	\sum_{k=0}^{N-1}(\omega^{i-j})^k=\sum_{k=0}^{N-1}1=N.
	\]
\end{proof}

Следовательно, обратное преобразование — это прямое FFT с корнем $\omega^{-1}$,
после чего нужно поделить все коэффициенты на $N$:
\[
a = \frac{1}{N}\,F_{\omega^{-1}}\,y.
\]

\subsubsection*{Сложность умножения многочленов через FFT}

Нужно:
\begin{itemize}
	\item одно прямое FFT для $A$,
	\item одно прямое FFT для $B$,
	\item покомпонентное умножение ($O(N)$),
	\item одно обратное FFT.
\end{itemize}
Итого $O(N\log N)$, где $N$ — ближайшая степень двойки, не меньшая $2n-1$.

При этом $2n-1 \le N < 2(2n-1) < 4n$, то есть $N=\Theta(n)$, поэтому итоговая сложность равна $O(n\log n)$.

\subsection*{Обобщение: NTT (Number-Theoretic Transform)}

Вместо $\mathbb{C}$ можно работать в кольце/поле $R$, если:
\begin{itemize}
	\item в $R$ существует элемент $\omega$ порядка $N$ (примитивный $N$-й корень из единицы);
	\item число $N$ обратимо в $R$ (нужно для деления на $N$ в обратном преобразовании).
\end{itemize}

\subsubsection*{Пример (поле $\mathbb{Z}_p$)}

Пусть $R=\mathbb{Z}_p$, где $p$ — простое, и
\[
N=2^k,\qquad p=c\cdot 2^k+1.
\]
Тогда $p-1$ кратно $N$, и существует первообразный корень $g$ по модулю $p$
(порождает мультипликативную группу порядка $p-1$).
Можно взять
\[
\omega \equiv g^c \pmod p,
\]
тогда $\omega$ имеет порядок $N$, и NTT работает полностью по модулю $p$.

\subsection*{Приложение: смысл коэффициентов DFT}

Пусть дан сигнал (последовательность)
\[
a_0,a_1,\dots,a_{N-1}.
\]
Определим дискретное преобразование Фурье:
\[
y_k=\sum_{t=0}^{N-1} a_t\,\omega^{kt},\qquad k=0,\dots,N-1,
\quad \text{где }\ \omega=e^{2\pi i/N}.
\]
Так как
\[
\omega^{kt}=\cos\!\left(\frac{2\pi k}{N}t\right)+
i\sin\!\left(\frac{2\pi k}{N}t\right),
\]
то
\[
\Re(y_k)=\sum_{t=0}^{N-1} a_t\cos\!\left(\frac{2\pi k}{N}t\right),
\qquad
\Im(y_k)=\sum_{t=0}^{N-1} a_t\sin\!\left(\frac{2\pi k}{N}t\right).
\]
Эти суммы — скалярные произведения сигнала с косинусом и синусом частоты $k$.
Если в сигнале сильно выражена гармоника частоты $k$, то вклады «складываются»,
и модуль $|y_k|$ получается большим; если такой частоты нет — вклады
в основном взаимно компенсируются и $|y_k|$ близок к нулю.
	